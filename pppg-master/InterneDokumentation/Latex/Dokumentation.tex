%% Erläuterungen zu den Befehlen erfolgen unter
%% diesem Beispiel.

\documentclass{scrartcl}

\usepackage[utf8]{inputenc}
\usepackage[T1]{fontenc}
\usepackage{lmodern}
%%\usepackage[ngerman]{babel}
\usepackage{amsmath}
\usepackage{fancyhdr}
\usepackage{graphicx} 
\usepackage{listings}
\lstset{basicstyle=\ttfamily}
\lstset{literate=%
	{Ö}{{\"O}}1
	{Ä}{{\"A}}1
	{Ü}{{\"U}}1
	{ß}{{\ss}}1
	{ü}{{\"u}}1
	{ä}{{\"a}}1
	{ö}{{\"o}}1
}

\pagestyle{fancy}
\fancyhf{} %kopf/fußzeile bereinigen

\fancyhead[L]{APP\\pppg}
\fancyhead[R]{Projektarbeit\\Dokumentation}
%\fancyhead[L]{1234}
%\includegraphics[width=0.3\textwidth]{\logo}\par\vspace{1cm} 
\vfill

\begin{document}
\section*{Entpacken des Archivs}
Das Archiv kann über folgenden Befehl entpackt werden:
\begin{lstlisting}
tar xvf pppg.tar
\end{lstlisting}
\section*{Quellcode übersetzen und Dokumentation erzeugen}
Durch den Aufruf von:
\begin{lstlisting}
ant
\end{lstlisting}
wird der Quelltext compiliert und die Dokumentation erzeugt. Anschließen steht die Datei pppg.jar zur Verfügung. Die Dokumentation steht im Ordner javadoc zur verfügung
\section*{Starten des Spieles}
\subsection*{Spielen auf dem lokalem Rechner}
\begin{lstlisting}
java -jar pppg.jar -red {human,random,simple,remote} 
-blue {human,random,simple,remote} -size {3,5,7,9,11}
\end{lstlisting}
Im Falle -red remote oder -blue remote, müssen -nameblue/-namered\{String\} und\newline
 -hostblue/-hostred\{String\} ergänzt werden
\subsection*{Spielen als Remote}
\begin{lstlisting}
java -jar pppg.jar --network -netname {String} 
-netplayer {human,random,simple} -nethost {host:port}
\end{lstlisting}
\subsection*{Optionale Argumente}
\begin{lstlisting}
-delay {>=0} --debug 
--win (deaktiviert alle externen Fenster außer der Graphik)
\end{lstlisting}
\subsection*{RMIRegistry starten}
\begin{lstlisting}
rmiregistry -J-Djava.rmi.server.codebase=file:
<Pfad zu benötigten Klassen>
\end{lstlisting}
\section*{Bedienung des Spiels}
Im unterem Bereich des Fensters sind verschiedene Informationen zum Aktuellem Spiel zu sehen.
Wird man nun aufgefordert einen Zug einzugeben, so muss man, je nach MoveType zwei Links oder eine Site und einen Link anklicken. Die Rheinfolge ist einzuhalten.
Unten ist auch zu sehen, welche Sites bereits ausgewählt sind. 
Beim Bewegen der Figur sind alle erreichbaren Site in hell Blau bzw. hell Rot eingefärbt.
\end{document}